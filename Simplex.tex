\documentclass{article}

\usepackage[utf8]{inputenc}
\usepackage{pifont}
\usepackage{amsmath}
\usepackage[spanish]{babel}
\title{Método Simplex}
\author{Yo Merengues}
\date{2020/03/11}
\setlength{\parindent}{0cm}
\begin{document}
\maketitle
\tableofcontents
\section{¿Qué es el método Simplex?}
Es un conjunto de métodos muy7 usados para resolver problemas de
programación lineal, en los cuales de alguna manera se busca el máximo
de una función lineal sobre un conjunto de variables que satisfaga un
conjunto de inecauaciones lineales.Fue creado por George Dantzig en 1947.
\section{Recomendaciones inciales}
\begin{itemize}
  \item Conviene escribir con $x_i$ con $i=1,2,...,n.$ $i\in N$ variables.
  \end{itemize}
  \subsection{Forma Estándar}
  \begin{equation*}
    \begin{aligned}
      \text{Maximizar} \quad & c^Tx\\
      \text{sujeto a} \quad &
      \begin{aligned}
      Ax &\leq b \\
      x_1,x_2,\ldots,x_n &\geq 0\\
    \end{aligned}
  \end{aligned}
\end{equation*}
Con $C^T$ Un Vector columna traspuesto, $\vec{x}=(x_1,x_2,\ldots,x_n)$
el vector de variables, $A$ una Matriz de mxn, y $b$ un vector columna
de mx1.
  \subsection{Forma Simplex}
  \begin{equation*}
    \begin{aligned}
      \text{Maximizar} \quad & c^Tx\\
      \text{sujeto a} \quad &
      \begin{aligned}
      Ax &=b \\
      x_1,x_2,\ldots,x_n &\geq 0\\
    \end{aligned}
  \end{aligned}
\end{equation*}
Con $C^T$ Un Vector columna traspuesto, $\vec{x}=(x_1,x_2,\ldots,x_n)$
el vector de variables, $A$ una Matriz de mxn, y $b$ un vector columna
de mx1.
\section{Método (Explicado con un ejemplo)}
  \subsection{Ejemplo:}
    \begin{equation*}
   \begin{aligned}
   \text{Maximizar} \quad & 3x+4y\\
   \text{sujeto a} \quad &
     \begin{aligned}
      x,y &\geq  0\\
      -x+y &\geq 2\\
     2x+y &\leq 4
     \end{aligned}
   \end{aligned}
 \end{equation*}
 \subsection{Primer Paso}
 \textbf{1.-} Pasar a su forma Estándar:

 Para ello 
 
 

 
\end{document}
